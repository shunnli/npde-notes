\chapter{绪论与预备知识}


\section{绪论}


对于控制方程——偏微分方程(组)进行初步分类:
\begin{itemize}
    \item 线性PDE:在方程中关于所有未知函数及其偏导数都是线性的,例如 $u_t + a(x,t) u_x = 0$;
    \item 非线性PDE:除了线性PDE之外的PDE,例如 $u_t + u u_x = 0$;
          拟线性PDE:在方程中关于所有未知函数的最高阶偏导数是线性的,例如 $u_t + u u_x = u_{xx}$。
\end{itemize}
常见模型方程:
\begin{itemize}
    \item 对流方程(双曲型方程)
    \item 波动方程(双曲型方程)
    \item 热方程(扩散方程)(抛物型方程)
    \item 泊松方程(抛物型方程)
    \item 对流扩散方程
    \item Euler 方程组
    \item N-S 方程组
    \item \dots
\end{itemize}


控制方程(组)不足以给出实际问题的唯一特解,需要加入一些条件来确定特解,这些条件称为定解条件,
常见的定解条件包括:边界条件、初值条件以及其它条件。
边界条件可以简单分为:
\begin{itemize}
    \item 周期性边界条件
    \item 非周期性边界条件
          \begin{itemize}
              \item[(a)] Dirichlet 边界条件
              \item[(b)] Neumann 边界条件
              \item[(c)] Robin 边界条件
          \end{itemize}
\end{itemize}
控制方程(组)加上适当的定解条件就得到了定解问题,常见的定解问题包括:
\begin{itemize}
    \item 初值问题(Cauchy问题):PDE(s) + 初值条件;
    \item 边值问题:PDE(s) + 边界条件;
    \item 初边值问题:PDE(s) + 初值条件 + 边界条件。
\end{itemize}


\begin{definition}
    称函数 $u$ 是定解问题的(经典)解,如果满足:
    \begin{itemize}
        \item 在区域 $\Omega$ 内部,$u$ 具有足够的光滑性以使得 PDE(s) 要求的各阶导数存在且连续,
              将 $u$ 和它的各阶导数代入PDE(s)中,可以使PDE(s)成为恒等式;
        \item 当 $\Omega$ 内部的点趋于边界 $\partial \Omega$ 时,
              定解条件所要求的 $u$ 以及它的各阶导数的极限处处存在,且使定解条件成立。
    \end{itemize}
\end{definition}

\begin{definition}
    称定解问题的解是稳定的,如果定解条件发生微小变化时,相应问题的解也只会发生微小变化,
    即:解对于定解条件存在连续依赖关系。
\end{definition}

\begin{definition}
    称定解问题是适定的,如果解存在唯一,并且关于定解条件是稳定的。
\end{definition}



我们只考虑针对适定问题的数值求解,求解偏微分方程数值解的主要步骤包括
\begin{enumerate}
    \item 时空区域剖分(离散)
    \item 偏微分方程(组)离散
    \item 定解条件(初值,边界条件)离散
    \item 求解离散方程组
\end{enumerate}
根据离散方式的不同,常见的数值方法可以简单分为如下几类:
\begin{itemize}
    \item 有限差分法
    \item 有限元方法
    \item 谱方法
    \item 其它方法
\end{itemize}
这里主要讨论有限差分法。



\section{预备知识}

\subsection*{差分算子}

考虑格点序列 $\{x_j\}_{j \in \mathbb{Z}}$,其中 $x_j = j \, \Delta x,\,(\Delta x > 0)$,对于定义在格点上的序列 $\{v_j\}_{j \in \mathbb{Z}}$,平移算子 $E^p$($p \in \mathbb{Z}$)定义为:
\[
    (E^p\,v)_j = v_{j+p}
\]
例如 $(E^{1}\,v)_j = v_{j+1}$, $(E^{0}\,v)_j = v_j$,$(E^{-1}\,v)_j = v_{j-1}$。
基于平移算子可以定义几种基本的差分算子:
\begin{itemize}
    \item 前差算子 $D_+ := \frac1{\Delta x}(E^1 - E^{0})$;
    \item 后差算子 $D_- := \frac1{\Delta x}(E^0 - E^{-1})$;
    \item 中心差算子 $D_0 := \frac1{2\Delta x}(E^1 - E^{-1})$。
\end{itemize}
它们都是对 $\frac{\partial}{\partial x}$ 的一阶或二阶近似:
\begin{align*}
    D_+\,e^{i \omega x_j} & = \frac1{\Delta x} (e^{i \omega \Delta x} -  1)\, e^{i \omega x_j} = (i \omega + \mathcal{O}(\omega^2 \Delta x)) \,e^{i \omega x_j}                        \\
    D_-\,e^{i \omega x_j} & = \frac1{\Delta x} (1 - e^{-i \omega \Delta x})\, e^{i \omega x_j} = (i \omega + \mathcal{O}(\omega^2 \Delta x)) \,e^{i \omega x_j}                        \\
    D_0\,e^{i \omega x_j} & = \frac1{2\Delta x} (e^{i \omega \Delta x} - e^{-i \omega \Delta x})\, e^{i \omega x_j} = (i \omega + \mathcal{O}(\omega^3 \Delta x^2))\, e^{i \omega x_j}
\end{align*}
高阶求导可以通过上述算子的乘积近似,例如定义 $D_+\,D_-$
\[
    (D_+\,D_- \,v)_j = (D_-\,D_+ \,v)_j =
    \frac{1}{\Delta x^2}(E^2\,v - 2E^1\,v + E^{-1}\,v)_j
    = \frac{1}{\Delta x^2}(v_{j+1} - 2v_{j} + v_{j-1})
\]
它是$\frac{\partial^2}{\partial x^2}$的二阶近似
\[
    D_+\,D_- \,e^{i \omega x_j}
    = \frac{1}{\Delta x^2}(e^{i \omega \Delta x}-2+e^{-i \omega \Delta x})\,e^{i \omega x_j}
    = (-\omega^2 + \mathcal{O}(\omega^4 \Delta x^2))\,e^{i \omega x_j}
\]
某些情况下还会使用不含 $\Delta x$ 的算子,例如
\begin{align*}
    \Delta_+ :={} & E^1 - E^{0}           \\
    \Delta_- :={} & E^0 - E^{-1}          \\
    \Delta_0 :={} & E^1 - E^{-1}          \\
    \delta :={}   & E^{1/2} - E^{-1/2}    \\
    \delta^2 :={} & E^1  - 2 E^0 + E^{-1}
\end{align*}
在二维和高维情况下会标记差分算子对应的维度以便于区分,例如
\begin{align*}
    \delta_x^2 \, v_{i,j} ={} & v_{i+1,j} - 2 v_{i,j} + v_{i-1,j} \\
    \delta_y^2 \, v_{i,j} ={} & v_{i,j+1} - 2 v_{i,j} + v_{i,j-1}
\end{align*}

\subsection*{有限维空间的内积和范数}

考虑 $m$ 维空间 $\mathbb{C}^m$,对应的内积和范数为
\[
    (u,v) = \sum_{j=1}^m \overline{u}_j v_j,\quad \|u\| = \sqrt{(u,u)}.
\]

\subsection*{周期性格点函数的内积和范数}

将$[a,b]$区间均分成$N+1$个等距子区间,对应$N+1$个节点,空间步长 $\Delta x = (b-a)/(N+1)$。
对于固定的 $N$ 或 $\Delta x$,在这些节点处的函数值属于 $\mathbb{C}^{N+1}$,
如果直接定义内积为
\[
    (u,v) = \sum_{j=0}^N \overline{u}_j v_j,
\]
在 $\Delta x \to 0$($N \to \infty$)时上式趋于无穷,无法有效刻画格点函数空间。
需要对内积的定义进行调整,定义离散的内积以及对应的范数为
\[
    (u,v)_{\Delta x} := \sum_{j=0}^N \overline{u}_j v_j \,\Delta x,
    \quad \|u\|_{\Delta x} := \sqrt{(u,u)_{\Delta x}}\,.
\]
显然有
\[
    (u,v)_{\Delta x} = \sum_{j=0}^N \overline{u}_j v_j \Delta x
    \quad \xrightarrow{N \to \infty} \quad
    \int_a^b \overline{u}(x) v(x)\,dx = (u,v)
\]
在下文中,很多情况下我们默认只讨论实值函数,因此可能会省略共轭记号。
