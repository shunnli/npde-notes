\chapter{Fourier基础}

\section{Fourier级数}

\begin{theorem}
    对于以 $2\pi$ 为周期的函数 $f \in C^1(\mathbb{R})$,对应的Fourier级数为
    \[
        S(x) = \frac{1}{\sqrt{2\pi}} \sum_{\omega = -\infty}^\infty \hat{f}(\omega) e^{i \omega x}
    \]
    其中 Fourier 系数为
    \[
        \hat{f}(\omega) = \frac{1}{\sqrt{2\pi}} \int_{0}^{2\pi} f(x)e^{-i \omega x} \,dx
    \]
    那么,Fourier级数 $S(x)$ 一致收敛于 $f(x)$。
\end{theorem}


\begin{theorem}
    对于 $2\pi$ 周期的分片 $C^1$ 函数 $f$,
    若 $f \in C^1((a,b))$,则在任意连续闭区间 $[\alpha,\beta] \subset (a,b)$,
    都有Fourier级数一致收敛于 $f(x)$。
    对于间断点 $x_0$,Fourier 级数收敛于 $\frac12(f(x_0^+) + f(x_0^-))$。
\end{theorem}



\begin{theorem}
    对于 $2\pi$ 周期的分片 $C^1$ 函数 $g(x)$,
    假设 $g$ 的 $p-1$ 阶导数连续,$p$ 阶导数是分片 $C^1$ 的,那么存在常数 $C$,使 $g$ 的Fourier系数满足
    \[
        |\hat{g}(\omega)| \le \frac{C}{|\omega|^{p+1}+1}
    \]
\end{theorem}

考虑内积和对应的 $L^2$ 范数
\[
    (f,g) := \int_0^{2\pi} \overline{f(x)}g(x)\,dx,\,\|f\| := \sqrt{(f,f)}
\]


\begin{lemma}
    指数函数 $\frac{1}{\sqrt{2\pi}} e^{i n x},n=0,\pm 1,\pm 2,\cdots$
    在 $L^2$ 内积下是标准正交的
    \[
        \left(\frac{1}{\sqrt{2\pi}} e^{i m x},\frac{1}{\sqrt{2\pi}} e^{i n x}\right)
        =\, \delta_{mn} =  \left\{
        \begin{aligned}
            0, m \neq n \\
            1, m = n
        \end{aligned}
        \right.\]
\end{lemma}


\begin{theorem}[Parseval 关系]
    对于 $2\pi$ 周期函数 $f,g \in L^2$
    \[
        f(x) = \frac{1}{\sqrt{2\pi}} \sum_{\omega = -\infty}^\infty \hat{f}(\omega) e^{i \omega x},\quad
        g(x) = \frac{1}{\sqrt{2\pi}} \sum_{\omega = -\infty}^\infty \hat{g}(\omega) e^{i \omega x}
    \]
    那么,
    \[
        (f,g) = \sum_{\omega = -\infty}^\infty \overline{\hat{f}(\omega)} \hat{g}(\omega),\quad
        \|f\|^2 = \sum_{\omega = -\infty}^\infty |\hat{f}(\omega)|^2
    \]
\end{theorem}




\begin{theorem}
    对于任意 $2\pi$ 周期函数 $f(x) \in L^2$,Fourier级数部分和在 $L^2$ 意义下收敛到 $f$,即
    \[
        \lim_{N \to \infty} \|f - S_N\| = 0.
    \]
    其中
    \[
        S_N(x) = \frac{1}{\sqrt{2\pi}} \sum_{\omega = -N}^N \hat{f}(\omega) e^{i \omega x},\quad
        \hat{f}(\omega) = \frac{1}{\sqrt{2\pi}} \int_0^{2\pi} e^{-i \omega x}f(x)\,dx.
    \]
\end{theorem}


\section{三角插值}

 {\color{blue}为了记号的方便,只讨论 $N$ 为偶数的情形。} 将 $[0,2\pi]$ 等分为 $N+1$ 个小区间,对应 $N+1$ 个节点:$x_j$, $j = 0,\cdots,N$,记空间步长 $\Delta x = \frac{2\pi}{N+1}$。
对于 $2\pi$ 周期的函数 $u(x)$,记 $u_j = u(x_j)$,
存在唯一的“三角函数”
\[
    \text{Int}_N\,u(x) = \frac{1}{\sqrt{2\pi}} \sum_{\omega = -N/2}^{N/2} \widetilde{u}(\omega) e^{i \omega x}
\]
满足 $\text{Int}_N\,u(x)$ 在 $\{x_j\}_{j=0,\dots,N}$ 插值 $u$,即
\[
    \text{Int}_N\,u(x_j) = \frac{1}{\sqrt{2\pi}} \sum_{\omega = -N/2}^{N/2} \widetilde{u}(\omega) e^{i \omega x_j} = u(x_j),\quad
    j=0,\cdots,N.
\]
下面默认 $u,v$ 均为 $2\pi$ 周期的函数。

\begin{lemma}
    指数函数 $\frac{1}{\sqrt{2\pi}} e^{i n x},n=0,\pm 1,\pm 2,\cdots$
    关于离散 $L^2$ 内积是标准正交的
    \[
        \left(\frac{1}{\sqrt{2\pi}} e^{i m x},\frac{1}{\sqrt{2\pi}} e^{i n x}\right)_{\Delta x}
        =\, \delta_{mn} =
        \begin{cases}
            0, & 0 < |m-n| \le N \\
            1, & m = n
        \end{cases}
    \]
\end{lemma}


\begin{theorem}
    三角插值 $\text{Int}_N\,u(x)$ 存在且唯一,具体表达式为
    \[
        \text{Int}_N\,u(x) = \frac{1}{\sqrt{2\pi}} \sum_{\omega = -N/2}^{N/2} \widetilde{u}(\omega) e^{i \omega x}
    \]
    其中的系数为
    \[
        \widetilde{u}(\omega) = \frac{1}{\sqrt{2\pi}}(e^{i \omega x},u)_{\Delta x},\quad
        |\omega| \le \frac{N}{2}
    \]
\end{theorem}


\begin{theorem}
    考虑对于 $u$ 和 $v$ 的三角插值 $\text{Int}_N\,u(x)$ 和 $\text{Int}_N\,v(x)$
    \[
        \text{Int}_N\,u(x) = \frac{1}{\sqrt{2\pi}} \sum_{\omega = -N/2}^{N/2} \widetilde{u}(\omega) e^{i \omega x},\quad
        \text{Int}_N\,v(x) = \frac{1}{\sqrt{2\pi}} \sum_{\omega = -N/2}^{N/2} \widetilde{v}(\omega) e^{i \omega x}.
    \]
    那么
    \[
        (u,v)_{\Delta x} = \sum_{\omega = -N/2}^{N/2} \overline{\widetilde{u}(\omega)} \widetilde{v}(\omega) = (\text{Int}_N\,u,\text{Int}_N\,v) = (\text{Int}_N\,u,\text{Int}_N\,v)_{\Delta x}.
    \]
    以及
    \[
        \|u\|_{\Delta x}^2 = \sum_{\omega = -N/2}^{N/2} |\widetilde{u}(\omega)|^2 = \|\text{Int}_N\,u\|^2 = \|\text{Int}_N\,u\|_{\Delta x}^2.
    \]
\end{theorem}


\begin{theorem}
    对于 $u(x)$ 和它的三角插值 $\text{Int}_N\,u(x)$,有如下不等式成立
    \[
        \| D^\ell_+\,u\|_{\Delta x}^2 \le
        \left\| \frac{d^\ell}{dx^\ell}\,\text{Int}_N \,u\right\|^2 \le
        \left(\frac{\pi}2\right)^{2\ell} \|D^\ell_+\,u\|_{\Delta x}^2,\quad \ell=1,2,\cdots
    \]
\end{theorem}

考虑 $u$ 的 Fourier 级数和三角插值 $\text{Int}_N\,u$
\[
    u(x) = \frac{1}{\sqrt{2\pi}} \sum_{\omega = -\infty}^\infty \widehat{u}(\omega) e^{i \omega x},\quad
    \text{Int}_N\,u(x) = \frac{1}{\sqrt{2\pi}} \sum_{\omega = -N/2}^{N/2} \widetilde{u}(\omega) e^{i \omega x}
\]
显然 Fourier 系数 $\widehat{u}(\omega)$ 和三角插值系数 $\widetilde{u}(\omega)$ 存在一定的联系。

\begin{lemma}
    Fourier 系数 $\widehat{u}(\omega)$ 和三角插值系数 $\widetilde{u}(\omega)$ 满足如下关系
    \[
        \widetilde{u}(\omega) = \sum_{\ell = -\infty}^\infty \widehat{u}(\omega + \ell(N+1)),\quad |\omega| \le \frac{N}2
    \]
\end{lemma}

特别地,如果对于 $|\omega| > \frac{N}2$ 都有 $\widehat{u}(\omega) = 0$,这表明 $u$ 不含高频部分,三角插值严格等于原函数:$\text{Int}_N\,u(x) = u(x)$,此时 $\widetilde{u}(\omega) = \widehat{u}(\omega)$,$|\omega| \le \frac{N}2$。

\begin{theorem}
    如果 $u$ 的 Fourier 系数满足
    \[
        |\widehat{u}(\omega)| \le \frac{C}{|\omega|^m+1},\quad m > 1
    \]
    那么
    \[
        \|u - \text{Int}_N\,u\|_{\Delta x} \le \frac{2C}{\sqrt{2\pi}}
        \left(\frac{N}2\right)^{1-m}
        \left(
        \frac{1}{m-1} + \frac{2(N+1)}N B_m
        \right)
    \]
    其中 $B_m$ 定义为 $B_m = \sum_{j = 1}^\infty \frac{1}{(2j-1)^m}$。
\end{theorem}

\begin{corollary}
    存在常数 $C_\ell > 0$,使得
    \[
        \left\|
        \frac{d^\ell \,u}{dx^\ell} - \frac{d^\ell \,\text{Int}_N\,u}{d x^\ell}
        \right\|_{\infty}
        \le C_\ell \left(\frac{N}2\right)^{1+\ell-m},\quad 1+\ell < m
    \]
\end{corollary}


\section{基于三角插值的收敛性定理的证明}



\begin{proof}[\normalfont\bfseries Proof of Theorem~\ref{thm:big-thm}]
    % Gustafsson, Bertil《Time Dependent Problems and Difference Methods》 。

    对于固定时刻 $t^n$,我们可以基于数值解 $\{v_j^n\}$ 获取对应时刻的三角插值多项式 $\text{Int}_N\,(v_j^n)$,具体表达式为
    \[
        \text{Int}_N\,(v_j^n)(x)  = \frac{1}{\sqrt{2\pi}} \sum_{\omega = -N/2}^{N/2} e^{i \omega  x} \widehat{Q}^n(\xi) \tilde{f}(\omega )
    \]
    在 $t^n$ 时刻的精确解 $u(x,t^n)$ 可以表示为
    \[
        u(x,t^n) = \frac{1}{\sqrt{2\pi}} \sum_{\omega = -\infty}^{\infty} e^{i \omega (x+t^n)} \hat{f}(\omega )
    \]
    取一个固定的整数 $M$ 使得 $0 < M < N/2$,那么
    {\small
    \begin{align*}
        \| u(\cdot,t^n) - \text{Int}_N\,(v_j^n) \|^2 ={} & \sum_{\omega = -N/2}^{N/2}
        \left| e^{i \omega t^n} \hat{f}(\omega ) - \widehat{Q}^n(\xi) \tilde{f}(\omega ) \right|^2
        + \sum_{|\omega| > N/2} |\hat{f}(\omega)|^2                                                                                                                          \\
        \le{}                                            & \sum_{\omega = -M}^{M} \left| e^{i \omega t^n} \hat{f}(\omega ) - \widehat{Q}^n(\xi) \tilde{f}(\omega ) \right|^2
        + 2 \sum_{|\omega| > M} |\hat{f}(\omega)|^2
        + 2 \sum_{|\omega| > M} |\widehat{Q}^n(\xi)|^2 |\tilde{f}(\omega)|^2
        \\
        =:{}                                             & I + I\!I + I\!I\!I
    \end{align*}}
    验证随着 $M,N \to \infty$,这三项的极限全部为 $0$,即可完成证明。

    $I\!I$ 最容易估计,由条件 (a) 可得
    \[
        \sum_{\omega = -\infty}^{\infty} |\hat{f}(\omega )|^2 < \infty
    \]
    因此
    \[
        \lim_{M \to \infty} I\!I = 2 \lim_{M \to \infty} \sum_{|\omega| > M} |\hat{f}(\omega)|^2 = 0
    \]

    $I\!I\!I$ 的估计依赖于条件 (a) 和 (b):由于
    \[
        \lim_{N \to \infty} \| \text{Int}_N\,f  - f \| = 0,
    \]
    以及
    \[
        \sup_{0 \le t_n \le T} |\widehat{Q}^n| \le K_s,
    \]
    因此
    \[
        \lim_{M \to \infty} I\!I\!I
        = 2 \lim_{M \to \infty} \sum_{|\omega| > M} |\widehat{Q}^n(\xi)|^2 |\tilde{f}(\omega)|^2
        \le
        4 K_s^2  \lim_{M \to \infty} \sum_{|\omega| > M}
        \Big(|\tilde{f}(\omega) - \hat{f}(\omega)|^2 + |\hat{f}(\omega)|^2 \Big) = 0.
    \]

    最后是 $I$ 的估计:对于固定的 $M$,利用条件 (a),(b) 和 (c) 可以得到
    \begin{align*}
        \lim_{N \to \infty} I ={} & \lim_{N \to \infty} \sum_{\omega = -M}^{M} \left| e^{i \omega t^n} \hat{f}(\omega ) - \widehat{Q}^n(\xi) \tilde{f}(\omega ) \right|^2 \\
        \le{}                     & 2 \lim_{N \to \infty} \sum_{\omega = -M}^{M}
        \left\{
        \left| \widehat{Q}^n(\xi) \left(\tilde{f}(\omega) - \hat{f}(\omega) \right) \right|^2
        + \left| \left(e^{i \omega t^n}  - \widehat{Q}^n(\xi)\right) \hat{f}(\omega ) \right|^2
        \right\} = 0
    \end{align*}

    因此,对于任意 $\varepsilon > 0$,可以选取足够大的 $M$ 使得 $I\!I + I\!I\!I < \varepsilon/2$,然后选取足够大的 $N$ 使得 $I < \varepsilon/2$,收敛性证毕。
\end{proof}
