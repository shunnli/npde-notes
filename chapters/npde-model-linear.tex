\chapter{模型方程:一维常系数线性对流/扩散方程}

\section{一维常系数线性对流/扩散方程的谐波解}

目前只考虑一维问题,采用周期性边界条件。考虑常系数线性对流方程
\[
    \left\{
    \begin{aligned}
         & u_t + a u_x = 0, \quad (x,t) \in \mathbb{R} \times \mathbb{R}^+, \\
         & u(x,0) = f(x), \quad x \in \mathbb{R},
    \end{aligned}
    \right.
\]
取初值 $f(x)$ 为 $2\pi$ 周期的,空间波数为 $\omega$ 的谐波(为了简化描述,在下文中我们均忽略谐波的常系数 $\frac{1}{\sqrt{2\pi}}$)
\[
    f(x) = e^{i \omega x} \hat{f}(\omega)
\]
记精确解为
\[
    u(x,t) = e^{i \omega x} \hat{u}(\omega,t)
\]
代入可得 $\hat{u}$ 满足的方程
\[
    \left\{
    \begin{aligned}
         & \hat{u}_t(\omega,t) = -i \omega a \hat{u}(\omega,t), \\
         & \hat{u}(\omega,0) = \hat{f}(\omega),
    \end{aligned}
    \right.
\]
解得
\[
    \hat{u}(\omega,t) = e^{- i \omega a t} \hat{f}(\omega),
    \quad \Rightarrow \quad
    u(x,t) = e^{i \omega a t} e^{- i \omega x} \hat{f}(\omega)
\]
对于一般的初值
\[
    f(x) = \sum_{\omega = -\infty}^{\infty} e^{i \omega x} \hat{f}(\omega)
\]
易得对应的精确解为
\[
    u(x,t) = \sum_{\omega = -\infty}^{\infty} e^{- i \omega a t} e^{i \omega x} \hat{f}(\omega),
    \quad
    \| u(x,t) \|^2 = \|u(x,0)\|^2
\]

考虑常系数线性扩散方程($b > 0$)
\[
    \left\{
    \begin{aligned}
         & u_t = b u_{xx}, \quad (x,t) \in \mathbb{R} \times \mathbb{R}^+, \\
         & u(x,0) = f(x), \quad x \in \mathbb{R},
    \end{aligned}
    \right.
\]
类似推导可得,对于一般的初值
\[
    f(x) = \sum_{\omega = -\infty}^{\infty} e^{i \omega x} \hat{f}(\omega)
\]
对应的精确解为
\[
    u(x,t) = \sum_{\omega = -\infty}^{\infty} e^{- \omega^2 b t} e^{i \omega x} \hat{f}(\omega),
    \quad
    \| u(x,t) \|^2 \le \|u(x,0)\|^2
\]


\section{经典格式及其分类}

直接对 PDE 中的时间导和空间导利用差商近似(前差,后差,中心差),
就可以得到很多经典格式。
例如考虑对流方程 $u_t + a u_x = 0$,可以直接得到很多经典格式:(默认 $t^n$ 为已知时间层,$t^{n+1}$ 为未知时间层)
\begin{enumerate}
    \item FTFS 格式
          \[
              \frac{v_{j}^{n+1} - v_{j}^n}{\Delta t} + a \frac{v_{j+1}^n - v_{j}^n}{\Delta x} = 0
          \]
    \item FTBS 格式
          \[
              \frac{v_{j}^{n+1} - v_{j}^n}{\Delta t} + a \frac{v_{j}^n - v_{j-1}^n}{\Delta x} = 0
          \]
    \item FTCS 格式
          \[
              \frac{v_{j}^{n+1} - v_{j}^n}{\Delta t} + a \frac{v_{j+1}^n - v_{j-1}^n}{2\Delta x} = 0
          \]
    \item BTFS 格式
          \[
              \frac{v_{j}^{n+1} - v_{j}^n}{\Delta t} + a \frac{v_{j+1}^{n+1} - v_{j}^{n+1}}{\Delta x} = 0
          \]
    \item BTBS 格式
          \[
              \frac{v_{j}^{n+1} - v_{j}^n}{\Delta t} + a \frac{v_{j}^{n+1} - v_{j-1}^{n+1}}{\Delta x} = 0
          \]
    \item BTCS 格式
          \[
              \frac{v_{j}^{n+1} - v_{j}^n}{\Delta t} + a \frac{v_{j+1}^{n+1} - v_{j-1}^{n+1}}{2\Delta x} = 0
          \]
    \item CTCS 格式(蛙跳格式)
          \[
              \frac{v_{j}^{n+1} - v_{j}^{n-1}}{2\Delta t} + a \frac{v_{j+1}^n - v_{j-1}^n}{2\Delta x} = 0
          \]
\end{enumerate}
对于扩散方程 $u_t = b u_{xx}$($b>0$),类似可得
\begin{enumerate}
    \item FTCS 格式
          \[
              \frac{v_{j}^{n+1} - v_{j}^n}{\Delta t} = b \frac{v_{j+1}^n - 2v_{j}^n + v_{j-1}^n}{\Delta x^2}
          \]
    \item BTCS 格式
          \[
              \frac{v_{j}^{n+1} - v_{j}^n}{\Delta t} = b \frac{v_{j+1}^{n+1} - 2v_{j}^{n+1} + v_{j-1}^{n+1}}{\Delta x^2}
          \]
    \item CTCS 格式
          \[
              \frac{v_{j}^{n+1} - v_{j}^{n-1}}{2\Delta t} = b \frac{v_{j+1}^{n} - 2v_{j}^{n} + v_{j-1}^{n}}{\Delta x^2}
          \]
\end{enumerate}

将已知时间层和未知时间层加权组合起来,就可以得到 $\theta$ 格式,例如对于扩散方程 $u_t = b u_{xx}$($b>0$)的 $\theta$ 格式为
\[
    \frac{v_{j}^{n+1} - v_{j}^n}{\Delta t}
    =
    b \left[
        \theta \frac{v_{j+1}^{n+1} - 2v_{j}^{n+1} + v_{j-1}^{n+1}}{\Delta x^2}
        + (1-\theta) \frac{v_{j+1}^n - 2v_{j}^n + v_{j-1}^n}{\Delta x^2}
        \right]
\]
其中 $0 \le \theta \le 1$ 对应 $t^{n+1}$ 层和 $t^n$ 层的权重。
取 $\theta = \frac12$ 得到的格式称为 Crank–Nicolson 格式。
考虑 $\theta$ 的临界值:取 $\theta = 0$ 得到 FTCS 格式,取 $\theta = 1$ 得到 BTCS 格式。


对这些差分格式有很多种分类方式:
\begin{itemize}
    \item 可以按照能否直接计算来分类:显格式,隐格式。
          显格式可以直接计算,而隐格式通常需要联列求解方程组,单步的计算成本通常更大。
          大部分显格式是条件稳定的,时间步长受到严格限制;大部分全隐格式是无条件稳定的,可以选取更大的时间步长。
    \item 可以按照涉及的时间层来分类:双层格式(单步格式),多层格式(多步格式)等。多层格式在启动时需要通过其它双层格式的辅助,例如CTCS格式可以使用FTCS格式来启动。
\end{itemize}

通过差商近似设计这些经典格式看起来非常自然,但是得到的这些格式是否未必满足我们的期望是不清楚的,
需要进一步分析差分格式的性质,包括相容性,稳定性和收敛性等。

\section{基于三角插值的收敛性定理}

考虑更一般的单步格式
\[
    v_j^{n+1} = Q\,v_j^n, \quad v_j^0 = f_j,\quad Q := \sum_{\mu = -r}^s A_\mu(\Delta t,\Delta x) E^\mu,
\]
我们可以在加上一些条件(相容,稳定等)之后,利用Fourier级数和三角插值证明双层格式的收敛性,证明需要如下几个条件:
\begin{enumerate}
    \item[(a)] 初值 $f$ 是分片连续的,可以展开为Fourier级数,
          \[
              f(x) = \frac{1}{\sqrt{2\pi}} \sum_{\omega = -\infty}^{\infty} e^{i \omega  x} \hat{f}(\omega ),\quad \sum_{\omega = -\infty}^{\infty} |\hat{f}(\omega )|^2 < \infty
          \]
          并且它的三角插值 $\text{Int}_N\,f$ 收敛于 $f$;
          \[
              \text{Int}_N\,f (x) = \frac{1}{\sqrt{2\pi}} \sum_{\omega = -N/2}^{N/2} e^{i \omega  x} \tilde{f}(\omega ),
              \quad \lim_{N \to \infty} \| \text{Int}_N\,f  - f \| = 0
          \]
    \item[(b)] 差分近似是稳定的,即存在常数 $K_s$ 使得对所有的 $\Delta t$ 和 $\Delta x$ 有:
          \[
              \sup_{0 \le t_n \le T} |\widehat{Q}^n| \le K_s
          \]
    \item[(c)] 差分近似是相容的,即对于每一个固定的 $\omega $,都有:($\xi = \omega \Delta x$)
          \[
              \lim_{\Delta t,\Delta x \to 0} \sup_{0 \le t^n \le T} |\widehat{Q}^n(\xi) - e^{i \omega  t_n}| = 0
          \]
\end{enumerate}

\begin{theorem}\label{thm:big-thm}
    对于有限时间区域 $0 \le t \le T$,考虑 $\Delta t,\Delta x \to 0$ 时,对于差分近似:
    \[
        v_j^{n+1} = Q\,v_j^n,\quad v_j^0 = f_j,\quad Q := \sum_{\mu = -r}^s A_\mu(\Delta t, \Delta x) E^\mu,
    \]
    若上述三个条件均成立,则差分近似解的三角插值 $\text{Int}_N\,(v_j^n)$ 收敛于偏微分方程的解 $u$,即:
    \[
        \lim_{\Delta t, \Delta x \to 0} \sup_{0 \le t^n \le T} \| u(\cdot,t^n) - \text{Int}_N\,(v_j^n)(\cdot) \| = 0
    \]
    其中 $\text{Int}_N\,(v_j^n)$ 定义为
    \[
        \text{Int}_N\,(v_j^n)(x)  = \frac{1}{\sqrt{2\pi}} \sum_{\omega = -N/2}^{N/2} e^{i \omega  x} \widehat{Q}^n(\xi) \tilde{f}(\omega )
    \]
\end{theorem}

\begin{proof}
    % Gustafsson, Bertil《Time Dependent Problems and Difference Methods》 。
    证明见附录。
\end{proof}

\section{相容性与局部截断误差}

为了刻画差分格式对偏微分方程的逼近程度,引入相容性的概念,相容性主要通过局部截断误差来体现。
将满足偏微分方程 $\mathcal{L} u = g$ 的充分光滑精确解 $u$ 代入差分格式 $L v_j^n = g_j^n$ 的两侧,
显然等号不再成立,记两侧的差值为局部截断误差:
\[
    \tau_j^n := L u_j^n - g_j^n
\]
我们希望局部截断误差是一个与时空网格相关的无穷小量,随着时空网格加密,局部截断误差收敛到 $0$。

注意在计算局部截断误差时,差分格式在形式上需要和偏微分方程的量纲保持对应,不要给差分格式两侧同时乘以 $\Delta t$ 或 $\Delta x$。例如
\[
    u_t + a u_x = g(x)
    \quad \sim \quad
    \frac{v_j^{n+1}-v_j^{n+1}}{2\Delta t} + a \frac{v_{j+1}^n-v_{j-1}^n}{2\Delta x} = g_j^n
\]
此时的局部截断误差为
\[
    \tau_j^n = \frac{u_j^{n+1}-u_j^{n+1}}{2\Delta t} + a \frac{u_{j+1}^n-u_{j-1}^n}{2\Delta x} - g_j^n
\]


对局部截断误差的计算主要使用泰勒展开即可,
常见的泰勒展开公式如下:时间一阶导近似
\begin{align*}
    \frac{u_j^{n+1}-u_j^n}{\Delta t} ={}      & (u_t)_j^n + \frac{\Delta t}2 (u_{tt})_j^n
    + \frac{\Delta t^2}6 (u_{ttt})_j^n + \mathcal{O}(\Delta t^3)                                                                              \\
    \frac{u_j^{n+1}-u_j^n}{\Delta t} ={}      & (u_t)_j^{n+1} - \frac{\Delta t}2 (u_{tt})_j^{n+1}
    + \frac{\Delta t^2}6 (u_{ttt})_j^{n+1} + \mathcal{O}(\Delta t^3)                                                                          \\
    \frac{u_j^{n+1}-u_j^n}{\Delta t} ={}      & (u_t)_j^{n+\frac12} + \frac{\Delta t^2}{24} (u_{ttt})_j^{n+\frac12} + \mathcal{O}(\Delta t^4) \\
    \frac{u_j^{n+1}-u_j^{n-1}}{2\Delta t} ={} & (u_t)_j^{n} + \frac{\Delta t^2}{6} (u_{ttt})_j^{n} + \mathcal{O}(\Delta t^4)
\end{align*}
空间一阶导近似
\begin{align*}
    \frac{u_{j+1}^{n}-u_j^n}{\Delta x} ={}        & (u_x)_j^n + \frac{\Delta x}2 (u_{xx})_j^n
    + \frac{\Delta x^2}6 (u_{xxx})_j^n + \mathcal{O}(\Delta x^3)                                                                                    \\
    \frac{u_{j}^{n}-u_{j-1}^n}{\Delta x} ={}      & (u_x)_j^{n} - \frac{\Delta x}2 (u_{xx})_j^{n}
    + \frac{\Delta x^2}6 (u_{xxx})_j^{n} + \mathcal{O}(\Delta x^3)                                                                                  \\
    \frac{u_{j+1}^{n}-u_j^n}{\Delta x} ={}        & (u_x)_{j+\frac12}^{n} + \frac{\Delta x^2}{24} (u_{xxx})_{j+\frac12}^n + \mathcal{O}(\Delta x^4) \\
    \frac{u_{j+1}^{n}-u_{j-1}^{n}}{2\Delta x} ={} & (u_x)_j^{n} + \frac{\Delta x^2}{6} (u_{xxx})_j^{n} + \mathcal{O}(\Delta x^4)
\end{align*}
空间二阶导近似
\[
    \frac{u_{j+1}^{n} - 2 u_j^n + u_{j-1}^n}{\Delta x^2} ={}
    (u_{xx})_{j}^n + \frac{\Delta x^2}{12} (u_{xxxx})_j^n + \mathcal{O}(\Delta x^4)
\]
平均值
\begin{align*}
    \frac{u_{j-1}^{n} + u_{j+1}^n}{2} ={} & u_j^n + \frac{\Delta x^2}2 (u_{xx})_j^n + \mathcal{O}(\Delta x^4)                     \\
    \frac{u_{j}^{n} + u_{j+1}^n}{2} ={}   & u_{j+\frac12}^n + \frac{\Delta x^2}8 (u_{xx})_{j+\frac12}^n + \mathcal{O}(\Delta x^4)
\end{align*}

通常在 $(x_j,t^n)$ 处泰勒展开得到的局部截断误差形如
\[
    \tau_j^n =  \mathcal{O}((\Delta x)^p + (\Delta t)^q)
\]
随着 $\Delta t,\Delta x \to 0$,显然有局部截断误差 $\tau_j^n \to 0$,此时称格式是无条件相容的。
但是也可能存在其它形式的结果,
如果必须假定 $\Delta x$ 和 $\Delta t$ 在趋于 $0$ 时满足一定约束,则称格式是有条件相容的。
例如 Lax–Friedrichs 格式是有条件相容的:需要固定 $r = \Delta t/\Delta x$,随着 $\Delta t \to 0$,
才有
\[
    \tau_j^n =  \mathcal{O}(\Delta t + \frac{\Delta x^2}{\Delta t}) =  \mathcal{O}(\Delta x)\to 0
\]

\begin{remark}
    泰勒展开的具体位置不会影响局部截断误差的主项,在不同位置展开的差异只是一个高阶小量。
\end{remark}

\begin{example}
    对于对流方程 $u_t + a u_x = 0$ 的 FTFS 格式,计算局部截断误差。
\end{example}

\begin{solution*}
    对于 FTFS 格式,将精确解 $u$ 代入格式两侧,在 $(x_j,t^n)$ 处展开即可
    \begin{align*}
        \tau_j^n ={} & \frac{u_j^{n+1}-u_j^n}{\Delta t} + a \frac{u_{j+1}^n - u_j^n}{\Delta x} \\
        ={}          & (u_t)_j^n + \frac{\Delta t}2 (u_{tt})_j^n + \mathcal{O}(\Delta t^3)
        + a \left[
            (u_x)_j^n + \frac{\Delta x}2 (u_{xx})_j^n + \mathcal{O}(\Delta x^3)
        \right]                                                                                \\
        ={}          & \mathcal{O}(\Delta x + \Delta t) \qedhere
    \end{align*}
\end{solution*}

\begin{example}
    对于扩散方程 $u_t = b u_{xx}$($b>0$)的 FTCS 格式,计算局部截断误差。
\end{example}

\begin{solution*}
    对于 FTCS 格式,将精确解 $u$ 代入格式两侧,在 $(x_j,t^n)$ 处展开即可
    \begin{align*}
        \tau_j^n ={} & \frac{u_j^{n+1}-u_j^n}{\Delta t} - b \frac{u_{j+1}^n - 2 u_j^n + u_{j-1}^n}{\Delta x^2} \\
        ={}          & (u_t)_j^n + \frac{\Delta t}2 (u_{tt})_j^n + \mathcal{O}(\Delta t^3)
        - b \left[
            (u_{xx})_{j}^n + \frac{\Delta x^2}{12} (u_{xxxx})_j^n + \mathcal{O}(\Delta x^4)
        \right]                                                                                                \\
        ={}          & \mathcal{O}(\Delta x^2 + \Delta t)  \qedhere
    \end{align*}
\end{solution*}


\begin{example}
    对于扩散方程 $u_t = b u_{xx}$($b>0$)的 $\theta$ 格式,计算局部截断误差。
\end{example}

\begin{solution*}
    对于 $\theta$ 格式,将精确解 $u$ 代入格式两侧,在 $(x_j,t^{n+\frac12})$ 处展开
    \begin{align*}
        \tau_j^{n+\frac12} ={} & \frac{u_j^{n+1}-u_j^n}{\Delta t} - b
        \left[
        \theta \frac{u_{j+1}^{n+1} - 2 u_j^{n+1} + u_{j-1}^{n+1}}{\Delta x^2}
        + (1-\theta) \frac{u_{j+1}^n - 2 u_j^n + u_{j-1}^n}{\Delta x^2}
        \right]                                                                                                                           \\
        ={}                    & (u_t)_j^{n+\frac12} + \frac{\Delta t^2}{24} (u_{ttt})_j^{n+\frac12} + \mathcal{O}(\Delta t^4)            \\
                               & - b \theta \left[
        (u_{xx})_{j}^{n+1} + \frac{\Delta x^2}{12} (u_{xxxx})_j^{n+1} + \mathcal{O}(\Delta x^4)
        \right]                                                                                                                           \\
                               & - b (1-\theta) \left[
        (u_{xx})_{j}^{n} + \frac{\Delta x^2}{12} (u_{xxxx})_j^{n} + \mathcal{O}(\Delta x^4)
        \right]
        \\
        ={}                    & -b \Delta t \left(\theta - \frac12\right) (u_{xxt})_j^{n+\frac12} + \mathcal{O}(\Delta x^2 + \Delta t^2)
    \end{align*}
    因此,当 $\theta = \frac12$ 即 Crank–Nicolson 格式时,局部截断误差为 $\mathcal{O}(\Delta x^2 + \Delta t^2)$;在其他情况下,局部截断误差为 $\mathcal{O}(\Delta x^2 + \Delta t)$。
\end{solution*}

\section{稳定性与放大因子}

稳定性关注满足差分格式的近似解在小扰动下的表现,假设提供一个 $\varepsilon$ 量级的初值小扰动(例如实际数值计算中的舍入误差等),
对数值解不能产生随时间指数级增长的影响,否则这个格式将不能应用于实际的数值计算中。

考虑双层差分格式
\[
    v_j^{n+1} = Q\,v_j^n,\quad v_j^0 = f_j,
\]
取谐波解初值为 $f_j = \hat{f}(\omega ) e^{i \omega  x_j}$,
将 $v_j^n =   \hat{v}^n(\omega ) e^{i \omega  x_j}$ 和 $v_j^{n+1} = \hat{v}^{n+1}(\omega ) e^{i \omega  x_j}$ 代入格式,
可以得到 $\hat{v}^n(\omega )$ 与 $\hat{v}^{n+1}(\omega )$ 之间的关系,定义放大因子(又称为增长因子)
\[
    \widehat{Q} = \widehat{Q}(\omega ) := \frac{\hat{v}^{n+1}(\omega )}{\hat{v}^n(\omega )}
\]
那么
\[
    \hat{v}^{n+1}(\omega ) = \widehat{Q}\, \hat{v}^n(\omega ) = \dots =  (\widehat{Q})^{n+1}\, \hat{v}^0(\omega ) = (\widehat{Q})^{n+1}\, \hat{f}(\omega )
\]
因此 $(\widehat{Q}(\omega))^{n}$ 体现了数值解 $v_j^n$ 随时间的变化,($L^2$ 模)稳定性主要关注对 $(\widehat{Q}(\omega))^{n}$ 的控制。

\begin{remark}
    这里从稳定性讨论直接引入放大因子是有点突兀,其实放大因子可以在更完善的理论基础上进行描述,具体可以参考张强《偏微分方程的有限差分方法》2.4节。
    上述公式与其说的放大因子的定义,不如说是放大因子的一种简便计算方法。
\end{remark}

\begin{definition}
    称双层差分格式 $v_j^{n+1} = Q\,v_j^n$ 是($L^2$ 模)稳定的,如果对应的放大因子 $\widehat{Q}$ 满足:
    \[
        \lim_{\Delta t, \Delta x \to 0} \sup_{0 \le t^n \le T} |(\widehat{Q}(\omega ))^n| \le K(T), \quad \forall\,\omega
    \]
\end{definition}

如果保证始终有 $|\widehat{Q}| \le 1$(严格的 von Neumann 条件),那么显然格式具有稳定性。
对于常系数对流方程 $u_t + a u_x = 0$ 的大部分格式,通常需要控制 $r = \frac{a \Delta t}{\Delta x}$ 来获得($L^2$ 模)稳定性,
如果一个格式要求 $|r| \le 1$ 才能保证 $|\widehat{Q}| \le 1$,那么称格式是条件稳定的,$|r| \le 1$ 就是稳定性条件。
如果对任意的 $r$ 均可保证 $|\widehat{Q}| \le 1$,则称格式是无条件稳定的。
对于常系数扩散方程 $u_t = b u_{xx}$($b > 0$)的大部分格式,则需要控制 $\mu = \frac{b \Delta t}{\Delta x^2}$ 来获得($L^2$ 模)稳定性。

\begin{remark}
    针对多层差分格式的放大因子和稳定性分析比双层差分格式更复杂,在~\ref{sec:multi-layer-format}~节会以蛙跳格式为例,进行专门的讨论。
\end{remark}

\begin{example}\label{eg:ftfs}
    对于对流方程 $u_t + a u_x = 0$ 的 FTFS 格式,计算放大因子并分析稳定性。
\end{example}

\begin{solution*}
    将 FTFS 格式整理为
    \[
        v_j^{n+1} = v_j^n - r \left( v_{j+1}^n - v_{j}^n \right), \quad r := \frac{a \Delta t}{\Delta x}
    \]
    因此
    \[
        \hat{v}^{n+1} = \hat{v}^n - r (e^{i \omega \Delta x} - 1) \hat{v}^n
    \]
    记 $\xi = \omega \Delta x$,得到放大因子
    \begin{gather*}
        \widehat{Q} ={}   1 - r (e^{i \xi} - 1)  = (1 + r - r \cos(\xi)) - i r \sin(\xi)     \\
        |\widehat{Q}|^2 ={} (1 + r - r \cos(\xi))^2 + r^2 \sin^2(\xi) = 1 + r(1+r) \xi^2 + \mathcal{O}(\xi^4)
    \end{gather*}
    因此:
    \begin{enumerate}
        \item 若 $a > 0$,则 $r > 0$,必然有 $|\widehat{Q}| > 1$,FTFS 格式不稳定;
        \item 若 $a < 0$,则 $r < 0$,由稳定性要求 $|\widehat{Q}| \le 1$ 可以推出 $-1 \le r < 0$,即 FTFS 格式有条件稳定。
    \end{enumerate}
\end{solution*}

\begin{example}\label{eg:btfs}
    对于对流方程 $u_t + a u_x = 0$ 的 BTFS 格式,计算放大因子并分析稳定性。
\end{example}

\begin{solution*}
    将 BTFS 格式整理为
    \[
        r v_{j+1}^{n+1} + (1-r) v_j^{n+1} = v_j^n, \quad r := \frac{a \Delta t}{\Delta x}
    \]
    因此
    \[
        \hat{v}^{n+1} = \frac{1}{1-r+r e^{i \omega \Delta x}}\hat{v}^n
    \]
    记 $\xi = \omega \Delta x$,得到放大因子
    \begin{gather*}
        \widehat{Q} ={}  \frac{1}{1 + r (e^{i \xi} - 1)} = \frac{1}{(1 - r + r \cos(\xi)) + i r \sin(\xi)}     \\
        |\widehat{Q}|^2 ={} \frac{1}{(1 - r + r \cos(\xi))^2 + r^2 \sin^2(\xi)} = 1 + r(1-r) \xi^2 + \mathcal{O}(\xi^4)
    \end{gather*}
    分情况讨论:
    \begin{enumerate}
        \item 若 $r=1$,则 $|\widehat{Q}| = 1$。
        \item 若 $r \in (0,1)$,则 $|\widehat{Q}| > 1$。
        \item 若 $r \in (-\infty,0) \cup (1,+\infty)$,则 $|\widehat{Q}| < 1$。
    \end{enumerate}
    因此:
    \begin{enumerate}
        \item 若 $a>0$,BTFS格式有条件稳定,稳定性要求为 $r \ge 1$;
        \item 若 $a<0$,BTFS格式无条件稳定。
    \end{enumerate}
\end{solution*}

\begin{example}\label{eg:ftcs}
    对于对流方程 $u_t + a u_x = 0$ 的 FTCS 格式,计算放大因子并分析稳定性。
\end{example}

\begin{solution*}
    将 FTCS 格式整理为
    \[
        v_j^{n+1} = v_j^n - \frac{r}2 \left( v_{j+1}^n - v_{j-1}^n \right), \quad r := \frac{a \Delta t}{\Delta x}
    \]
    因此
    \[
        \hat{v}^{n+1} = \hat{v}^n - \frac{r}2 (e^{i \omega \Delta x} - e^{-i \omega \Delta x}) \hat{v}^n
    \]
    记 $\xi = \omega \Delta x$,得到放大因子
    \begin{gather*}
        \widehat{Q} ={}  1 - \frac{r}2 (e^{i \xi} - e^{-i \xi}) = 1 -  i r \sin(\xi)            \\
        |\widehat{Q}|^2 ={} 1 + r^2 \sin^2(\xi) \ge 1
    \end{gather*}
    因此 FTCS 格式不稳定。
\end{solution*}

\begin{example}\label{eg:btcs}
    对于对流方程 $u_t + a u_x = 0$ 的 BTCS 格式,计算放大因子并分析稳定性。
\end{example}

\begin{solution*}
    将 BTCS 格式整理为
    \[
        v_j^{n+1} + \frac{r}2 \left( v_{j+1}^{n+1} - v_{j-1}^{n+1} \right) = v_j^n, \quad r := \frac{a \Delta t}{\Delta x}
    \]
    因此
    \[
        \hat{v}^{n+1} = \frac{1}{1 + \frac{r}2 (e^{i \omega \Delta x} - e^{-i \omega \Delta x})} \hat{v}^n
    \]
    记 $\xi = \omega \Delta x$,得到放大因子
    \begin{gather*}
        \widehat{Q} ={} \frac{1}{1 + \frac{r}2 (e^{i \xi} - e^{-i \xi})} = \frac{1}{1 + i r \sin(\xi)}            \\
        |\widehat{Q}|^2  ={} \frac{1}{1 + r^2 \sin^2(\xi)} \le 1
    \end{gather*}
    因此 BTCS 格式无条件稳定。
\end{solution*}

\begin{example}
    对于扩散方程 $u_t = b u_{xx}$($b>0$)的 FTCS 格式,计算放大因子并分析稳定性。
\end{example}

\begin{solution*}
    将 FTCS 格式整理为
    \[
        v_j^{n+1} = v_j^n + \mu \left( v_{j+1}^n - 2 v_j^n + v_{j-1}^n \right), \quad \mu := \frac{b \Delta t}{\Delta x^2}
    \]
    因此
    \[
        \hat{v}^{n+1} = \hat{v}^n + \mu (e^{i \omega \Delta x} - 2 + e^{- i \omega \Delta x}) \hat{v}^n
    \]
    记 $\xi = \omega \Delta x$,得到放大因子
    \begin{gather*}
        \widehat{Q} ={} 1 + \mu (e^{i \xi} - 2 + e^{- i \xi}) =  1 - 4 \mu \sin^2(\frac{\xi}2) \\
        |\widehat{Q}| ={} \left|1 - 4 \mu \sin^2(\frac{\xi}2)\right|
    \end{gather*}
    由稳定性要求 $|\widehat{Q}| \le 1$ 可以推出 $0 < \mu \le \frac12$。
    因此 FTCS 格式有条件稳定,具体的稳定性要求为
    \[
        \mu = \frac{b \Delta t}{\Delta x^2} \le \frac12 \qedhere
    \]
\end{solution*}

\begin{example}
    对于扩散方程 $u_t = b u_{xx}$($b>0$)的 BTCS 格式,计算放大因子并分析稳定性。
\end{example}

\begin{solution*}
    将 BTCS 格式整理为
    \[
        v_j^{n+1} - \mu \left( v_{j+1}^{n+1} - 2 v_j^{n+1} + v_{j-1}^{n+1} \right) = v_j^n, \quad \mu := \frac{b \Delta t}{\Delta x^2}
    \]
    因此
    \[
        \hat{v}^{n+1} - \mu (e^{i \omega \Delta x} - 2 + e^{- i \omega \Delta x}) \hat{v}^{n+1} = \hat{v}^n
    \]
    记 $\xi = \omega \Delta x$,得到放大因子
    \begin{gather*}
        \widehat{Q} ={} \frac{1}{1 - \mu (e^{i \xi} - 2 + e^{- i \xi})} =  \frac{1}{1 + 4 \mu \sin^2(\frac{\xi}2)} \\
        |\widehat{Q}| ={} \frac{1}{1 + 4 \mu \sin^2(\frac{\xi}2)} \le 1
    \end{gather*}
    始终满足稳定性要求 $|\widehat{Q}| \le 1$,因此 BTCS 格式无条件稳定。
\end{solution*}


\section{收敛性与整体误差}

我们自然希望随着时空网格的加密,满足差分格式的数值解会收敛到满足偏微分方程的精确解,格式的收敛性主要通过整体误差来刻画。
定义整体误差为精确解和数值解之差
\[
    e_j^n = u(x_j,t^n) - v_j^n,\quad E^n = \max_j(|e_j^n|).
\]
收敛性要求随着 $\Delta x,\Delta t \to 0$,有$E^n \to 0$。

\begin{example}
    对于扩散方程 $u_t = b u_{xx}$($b>0$)的 FTCS 格式,分析其收敛性。
\end{example}

\begin{solution*}
    记 $\mu = \frac{b \Delta t}{\Delta x^2}$,
    数值解 $v_j^n$ 满足
    \[
        v_j^{n+1} = v_j^n + \mu (v_{j+1}^n - 2 v_j^n + v_{j-1}^n),
    \]
    精确解 $u_j^n$ 满足($u_j^n = u(x_j,t^n)$)
    \[
        u_j^{n+1} = u_j^n + \mu (u_{j+1}^n - 2 u_j^n + u_{j-1}^n) + \Delta t\, \tau_j^n.
    \]
    其中 $\tau_j^n = \mathcal{O}(\Delta x^2 + \Delta t)$。
    定义误差
    \[
        e_j^n = u(x_j,t^n) - v_j^n,\quad E^n = \max_j(|e_j^n|).
    \]
    在满足稳定性要求 $\mu \le \frac12$ 时,存在常数 $A>0$,使得误差满足
    \begin{align*}
        e_j^{n+1} ={}   & e_j^n + \mu (e_{j+1}^n - 2 e_j^n + e_{j-1}^n) + \Delta t\, \tau_j^n          \\
        |e_j^{n+1}| ={} & (1-2\mu) |e_j^n| + \mu |e_{j+1}^n| + \mu |e_{j-1}^n| + \Delta t\, |\tau_j^n| \\
        \le{}           & E^n + \Delta t\, A(\Delta x^2 + \Delta t)                                    \\
        E^{n+1} \le{}   & E^n + \Delta t\, A(\Delta x^2 + \Delta t)                                    \\
        \le{}           & \dots \le E^0 + (n+1)\Delta t\, A(\Delta x^2 + \Delta t)
    \end{align*}
    其中 $E^0 = 0$。随着 $\Delta x,\Delta t \to 0$,$(n+1) \Delta t \to t^*$,有
    \[
        E^{n+1} \le t^* A(\Delta x^2 + \Delta t) \to 0, \quad (\Delta x,\Delta t \to 0)
    \]
    得证 FTCS 格式的收敛性。
\end{solution*}





\section{CFL 条件与迎风格式}

对于对流方程,需要考虑数值解依赖区和精确解依赖区的关系,对于对流方程 $u_t + a u_x = 0$,具有特征线,
特征线满足方程
\[
    \frac{d x(t)}{dt} = a, \quad \Rightarrow \quad x - a t = \text{constant}.
\]
由于
\[
    \frac{d}{dt} u(x(t),t) = u_x(x(t),t) \frac{d x(t)}{dt} + u_t(x(t),t) = a u_x + u_t = 0
\]
因此精确解沿着特征线保持不变。

考虑任意点 $P(x,t)$ 处的精确解 $u(x,t)$,根据特征线的性质,$u(x,t)$ 由过该点的特征线与 $t=0$ 交点 $P_0(x_0,0)$ 的初值决定
\[
    u(x,t) = u_0(x_0) = u_0(x - a t),\quad (x_0 = x - a t)
\]
称 $D_p = \{P_0\} = \{(x_0,0)\} = \{(x - a t,0)\}$ 为精确解 $u(x,t)$ 的依赖区。

对于数值格式,称数值解 $v_j^n$ 的依赖区为计算 $v_j^n$ 的过程中涉及到的空间区域 $N_p$。
例如考虑 $u_t + a u_x = 0$ 的 FTBS 格式
\[
    v_{j}^{n+1} = v_j^n - \frac{a \Delta t}{\Delta x} \left(v_{j}^n - v_{j-1}^n\right)
\]
则 $v_j^n$ 的计算涉及到的空间区域为 $N_p = [x_{j-n},\, x_j]$。

\begin{definition}[Courant–Friedrichs–Lewy condition]
    称保证如下关系成立的条件为 CFL 条件
    \[
        \text{精确解依赖区}\,D_p \,\subset\, \text{数值解依赖区}\,N_p
    \]
\end{definition}

\begin{remark}
    通常可以基于上述定义推出更具体的形如 $\Delta t \le C \Delta x$ 的时间步长限制。
\end{remark}

\begin{theorem}
    CFL 条件是差分格式收敛的必要条件。
\end{theorem}

\begin{remark}
    CFL 条件也适用对变系数和非线性双曲问题的数值格式分析,但只是这些格式收敛的必要条件。
\end{remark}

\begin{example}\label{eg:upwind}
    对于方程 $u_t + a u_x = 0$:
    \begin{itemize}
        \item 当 $a > 0$ 时,FTBS 格式有条件收敛,CFL 条件是 $a \Delta t \le \Delta x$,FTFS 格式不收敛;
        \item 当 $a < 0$ 时,FTFS 格式有条件收敛,CFL 条件是 $a \Delta t \le \Delta x$,FTBS 格式不收敛;
    \end{itemize}
\end{example}

\begin{solution*}
    根据对称性,只考虑 $a>0$ 的情形。
    精确解 $u(x_j,t^n)$ 的依赖区为 $D_p = \{(x_j - a t^n,0)\}$。
    对于 FTBS 格式,$v_j^n$ 的数值解依赖区为 $N_p = [x_{j-n},\, x_{j}]$,
    根据 CFL 条件可得
    \[
        x_j - a t^n \in [x_{j-n},\, x_{j}],
        \quad \Rightarrow \quad a \Delta t \le \Delta x.
    \]
    对于 FTFS 格式,$v_j^n$ 的数值解依赖区为 $N_p = [x_{j},\, x_{j+n}]$,此时 $D_p$ 不可能被包含在 $N_p$ 中,因此不可能收敛。
\end{solution*}



对于对流方程,可以根据特征线的走向(风向)来定义迎风格式:
\begin{itemize}
    \item 迎风格式:风向与模板方向一致的差分格式。
    \item 逆风格式:风向与模板方向不一致的差分格式。
\end{itemize}
具体来说,对于 $u_t + a u_x = 0$:
\begin{itemize}
    \item 在 $a > 0$ 时,FTBS 格式为迎风格式,FTFS 格式为逆风格式。
    \item 在 $a < 0$ 时,FTFS 格式为迎风格式,FTBS 格式为逆风格式。
\end{itemize}
也可以将迎风格式写为如下形式(记 $r = a \Delta t / \Delta x$)
\[
    v_j^{n+1} = v_j^n - \frac{r}2 (v_{j+1}^n - v_{j-1}^{n}) + \frac{|r|}2 (v_{j+1}^n - 2 v_j^n + v_{j-1}^{n})
    =
    \begin{cases}
        v_j^n - r (v_{j}^n - v_{j-1}^{n}), & a > 0 \\
        v_j^n - r (v_{j+1}^n - v_{j}^{n}), & a < 0
    \end{cases}
\]
由例~\ref{eg:upwind} 可知,迎风格式是条件稳定的,逆风格式则是无条件不稳定的。
例~\ref{eg:ftfs} 通过计算放大因子得到的($L^2$ 模)稳定性结论与这里相符。

\begin{remark}
    这里主要讨论的是显格式,因为显格式通常是局部更新,而隐格式通常是全局联列求解,并不能直接套用上述逻辑。
    如果把时间变量取负,可以得到显式迎风/逆风格式与“隐式迎风/逆风格式”之间的对称性关系:(推导过程参考例~\ref{eg:btfs},整理结果见附录)
    \begin{enumerate}
        \item 显式迎风格式有条件稳定,显式逆风格式无条件不稳定;
        \item “隐式迎风格式”有条件稳定,“隐式逆风格式”无条件稳定;
    \end{enumerate}
    因此,迎风性质可能对隐格式起到“反作用”。
\end{remark}

\section{不稳定格式的修正}

对于对流方程 $u_t + a u_x = 0$ 直接设计的 FTCS 格式
\[
    \frac{v_j^{n+1}-v_j^n}{\Delta t} + a \frac{v_{j+1}^n-v_{j-1}^n}{2\Delta x} = 0
\]
虽然 FTCS 格式具有局部截断误差 $\mathcal{O}(\Delta x^2 + \Delta t)$,但是例~\ref{eg:ftcs} 的分析表明它是无条件不稳定的。

\begin{remark}
    差分格式的相容性和稳定性之间没有必然的联系,设计一个好的差分格式不仅要考虑相容性,更要考虑稳定性。
\end{remark}

考虑给对流方程的右端加入人工黏性项
\[
    u_t + a u_x =  \sigma \Delta x\, u_{xx},\quad (\sigma \ge 0)
\]
离散得到
\[
    \frac{v_j^{n+1}-v_j^n}{\Delta t} + a \frac{v_{j+1}^n-v_{j-1}^n}{2\Delta x} = \sigma \Delta x \frac{v_{j+1}^n-2 v_j^n+v_{j-1}^n}{\Delta x^2}
\]
取两个特殊的 $\sigma$ 可以得到条件稳定的 Lax–Friedrichs 和 Lax-Wendroff 格式:

\begin{itemize}
    \item[(a)] Lax–Friedrichs 格式:取$\sigma = \frac{\Delta x}{2\,\Delta t}$,可以得到
          \[
              \frac{v_j^{n+1}-v_j^n}{\Delta t} + a \frac{v_{j+1}^n-v_{j-1}^n}{2\Delta x} = \left(\frac{\Delta x^2}{2\Delta t}\right) \frac{v_{j+1}^n-2 v_j^n+v_{j-1}^n}{\Delta x^2}
          \]
          计算可得局部截断误差为$\mathcal{O}(\Delta t + \frac{\Delta x^2}{\Delta t})$,因此是一个有条件相容的格式。
          整理可得
          \[
              \frac{v_j^{n+1}-\frac12(v_{j-1}^n+v_{j+1}^n)}{\Delta t} + a \frac{v_{j+1}^n-v_{j-1}^n}{2\Delta x} = 0
          \]
          这也可以解释为:将 $v_j^n$ 替换为空间上的平均 $\frac12(v_{j-1}^{n} + v_{j+1}^{n})$。

    \item[(b)] Lax-Wendroff格式:取$\sigma = \frac{a^2 \Delta t}{2\,\Delta x}$,可以得到
          \[
              \frac{v_j^{n+1}-v_j^n}{\Delta t} + a \frac{v_{j+1}^n-v_{j-1}^n}{2\Delta x} = \left(\frac{a^2 \Delta t}{2}\right) \frac{v_{j+1}^n-2 v_j^n+v_{j-1}^n}{\Delta x^2}
          \]
          计算可得局部截断误差为$\mathcal{O}(\Delta x^2+\Delta t^2)$。整理可得
          \[
              v_{j}^{n+1} = v_j^n + \Delta t \,(-a)\,\left(\frac{v_{j+1}^n- v_{j-1}^n}{2\Delta x}\right)
              + \frac{\Delta t^2}2\,a^2\,\left(\frac{v_{j+1}^n-2v_{j}^n+v_{j-1}^n}{\Delta x^2}\right)
          \]
          这也可以解释为:对精确解 $u$ 关于时间进行二阶泰勒展开的近似
          \begin{align*}
              u(x_j,t^{n+1})
               & \approx u(x_j,t^n) + \Delta t \,u_t(x_j,t^n) + \frac{\Delta t^2}2\,u_{tt}(x_j,t^n)      \\
               & = u(x_j,t^n) + \Delta t \,(-a)\,u_x(x_j,t^n) + \frac{\Delta t^2}2\,a^2\,u_{xx}(x_j,t^n)
          \end{align*}
\end{itemize}

\begin{remark}
    虽然都可以解释为添加人工黏性项,但是 Lax–Friedrichs 格式和 Lax-Wendroff 格式的各种性质很不相同:局部截断误差,耗散色散,是否产生数值振荡等。
\end{remark}

对于扩散方程 $u_t = b u_{xx}$ 直接设计的CTCS格式(Richardson格式)
\[
    \frac{v_j^{n+1}-v_{j-1}^n}{2\Delta t} = b \frac{v_{j+1}^n - 2 v_j^n + v_{j-1}^n}{\Delta x^2}
\]
虽然它具有局部截断误差 $\mathcal{O}(\Delta x^2+\Delta t^2)$,但是它是无条件不稳定的。
考虑将 $v_j^n$ 替换为时间平均 $\frac12(v_j^{n-1} + v_j^{n+1})$,得到 Du Fort-Frankel 格式
\[
    \frac{v_j^{n+1}-v_{j-1}^n}{2\Delta t} = b \frac{v_{j+1}^n - (v_{j}^{n-1}+v_j^{n+1}) + v_{j-1}^n}{\Delta x^2}
\]
Du Fort-Frankel 格式的局部截断误差为
\[
    \mathcal{O}(\Delta x^2 + \Delta t^2 + \frac{\Delta t^2}{\Delta x^2}).
\]
因此是一个有条件相容的格式。虽然DF格式是显格式,却是无条件稳定的。


\begin{example}
    对于对流方程 $u_t + a u_{x} = 0$ 的 Lax-Friedrichs 格式,计算放大因子并分析稳定性。
\end{example}

\begin{solution*}
    将 Lax-Friedrichs 格式整理为
    \[
        v_{j}^{n+1} = \frac{1}2 (v_{j-1}^n + v_{j+1}^n) - \frac{r}2 (v_{j+1}^n - v_{j-1}^n).
        \quad r = \frac{a \Delta t}{\Delta x}.
    \]
    因此
    \[
        \hat{v}^{n+1} = \left[
            \frac12 (e^{i \omega \Delta x} + e^{- i \omega \Delta x})
            - \frac{r}2 (e^{i \omega \Delta x} - e^{- i \omega \Delta x})
            \right] \hat{v}^n
    \]
    记 $\xi = \omega \Delta x$,得到放大因子
    \begin{align*}
        \widehat{Q}
                        & = \frac12 (e^{i \xi} + e^{- i \xi})
        - \frac{r}2 (e^{i \xi} - e^{- i \xi})
        \\
                        & = \cos(\xi) - i r \sin(\xi)
        \\
        |\widehat{Q}|^2 & =
        1 - (1-r^2) \sin^2(\xi)
    \end{align*}
    因此当且仅当 $0 < r \le 1$ 时,$|\widehat{Q}| \le 1$,Lax-Friedrichs 格式具有稳定性。
\end{solution*}

\begin{example}
    对于对流方程 $u_t + a u_{x} = 0$ 的 Lax-Wendroff 格式,计算放大因子并分析稳定性。
\end{example}

\begin{solution*}
    将 Lax-Wendroff 格式整理为
    \[
        v_{j}^{n+1}
        = v_j^n - \frac{r}2(v_{j+1}^n- v_{j-1}^n) + \frac{r^2}2
        (v_{j+1}^n-2v_{j}^n+v_{j-1}^n),
        \quad r = \frac{a \Delta t}{\Delta x}.
    \]
    因此
    \[
        \hat{v}^{n+1} = \left[
            1 - \frac{r}2(e^{i \omega \Delta x} - e^{- i \omega \Delta x})
            + \frac{r^2}2 (e^{i \omega \Delta x} - 2 + e^{- i \omega \Delta x})
            \right] \hat{v}^n
    \]
    记 $\xi = \omega \Delta x$,得到放大因子
    \begin{align*}
        \widehat{Q}
                        & = 1 - \frac{r}2(e^{i \xi} - e^{- i \xi})
        + \frac{r^2}2 (e^{i \xi} - 2 + e^{- i \xi})
        \\
                        & =
        1 -  i r \sin(\xi) + \frac{r^2}2 (-4 \sin^2(\frac{\xi}2))
        \\
                        & = 1 - 2r^2 \sin^2(\frac{\xi}2)
        - i r \sin(\xi)                                            \\
        |\widehat{Q}|^2 & =
        1 - 4 r^2 (1-r^2) \sin^4(\frac{\xi}2)
    \end{align*}
    因此当且仅当 $0 < r \le 1$ 时,$|\widehat{Q}| \le 1$,Lax-Wendroff 格式具有稳定性。
\end{solution*}


\section{其它方法构造差分格式}

可以通过 PDE 的积分形式来构造差分格式,未知量可以是格点处的函数值或函数在单元的平均值,选取合适的控制体进行数值积分即可。
在构造时使用的数值积分公式精度会直接决定差分格式的精度,常用的数值积分公式包括
\begin{align*}
    \int_{a}^{b} f(x)\,dx & = (b-a) f(a) + \frac{(b-a)^2}4 f'(\xi)                     \\
    \int_{a}^{b} f(x)\,dx & = \frac{(b-a)}{2}(f(a)+f(b)) - \frac{(b-a)^3}{12} f''(\xi) \\
    \int_{a}^{b} f(x)\,dx & = (b-a) f(\frac{a+b}2) + \frac{(b-a)^3}{24} f''(\xi)
\end{align*}
尤其注意在应用时的积分余项形式,例如
\begin{align*}
    \int_{x_{j-\frac12}}^{x_{j+\frac12}} u^{n+1} - u^n \,dx
    ={} & \Delta x (u_j^{n+1} - u_j^n) + \frac{\Delta x^3 }{24} \Big((u_{xx})^{n+1} - (u_{xx})^n\Big)\Big|_{x=\xi}       \\
    ={} & \Delta x (u_j^{n+1} - u_j^n) +  \mathcal{O}(\Delta x^3 {\color{red}\Delta t})                                  \\
    \int_{t^n}^{t^{n+1}} (u_{j+1} - u_{j-1})\, dt
    ={} & \Delta t (u_{j+1}^n - u_{j-1}^n) + \frac{\Delta t^2 }{4} \Big((u_{t})_{j+1} - (u_{t})_{j-1}\Big)\Big|_{t=\eta} \\
    ={} & \Delta t (u_{j+1}^n - u_{j-1}^n) + \mathcal{O}(\Delta t^2 {\color{red}\Delta x}).
\end{align*}

\begin{example}
    对于对流方程 $u_t + a u_x = 0$,使用积分方法推导一个差分格式,要求局部截断误差为 $\mathcal{O}(\Delta x^2+\Delta t^2)$。
\end{example}

\begin{solution*}
    取时空控制体 $\Omega_j^n = [x_{j-1},x_{j+1}] \times [t^n,t^{n+1}]$,那么
    \begin{align*}
        \int_{\Omega_j^n} u_t\,dt\,dx ={}                        & - a \int_{\Omega_j^n} u_x\,dx\,dt \\
        \int_{x_{j-1}}^{x_{j+1}} u(x,t^{n+1}) - u(x,t^n)\,dx ={} &
        - a \int_{t^n}^{t^{n+1}} u(x_{j+1},t) - u(x_{j-1},t) \,dt
    \end{align*}
    使用数值积分公式可得
    \begin{align*}
        \text{LHS} ={} & 2 \Delta x (u_j^{n+1} - u_j^n) +  \mathcal{O}(\Delta x^3 \Delta t) \\
        \text{RHS} ={} & - a \Delta t
        \left[
            \frac12 (u_{j+1}^n - u_{j-1}^n) + \frac12 (u_{j+1}^{n+1} - u_{j-1}^{n+1})
            \right] + \mathcal{O}(\Delta t^3 \Delta x)
    \end{align*}
    整理得到
    \[
        \frac{u_{j}^{n+1} - u_j^n}{\Delta t}
        + a \left[ \frac12 \frac{u_{j+1}^n - u_{j-1}^n}{2\Delta x} + \frac12 \frac{u_{j+1}^{n+1} - u_{j-1}^{n+1}}{2\Delta x} \right]
        = \mathcal{O}(\Delta x^2 + \Delta t^2)
    \]
    对应的差分格式为
    \[
        \frac{v_{j}^{n+1} - v_j^n}{\Delta t}
        + a \left[ \frac12 \frac{v_{j+1}^n - v_{j-1}^n}{2\Delta x} + \frac12 \frac{v_{j+1}^{n+1} - v_{j-1}^{n+1}}{2\Delta x} \right] = 0 \qedhere
    \]
\end{solution*}



还可以直接指定离散模板,通过待定系数法构造数值微分/积分公式,进而构造差分格式。

\begin{example}
    针对方程 $u_t + a u_x = 0$,构造如下模板的差分格式
    \[
        v_j^{n+1} = \alpha v_{j-1}^n + \beta v_j^n + \gamma v_{j+1}^n
    \]
    调整系数使其达到最优的局部截断误差阶。
\end{example}

\begin{solution*}
    在 $(x_j,t^n)$ 处展开以计算局部截断误差
    \begin{align*}
        \Delta t \tau_j^n ={} & u_j^{n+1} - \alpha u_{j-1}^n - \beta u_j^n - \gamma u_{j+1}^n                                \\
        ={}                   & u_j^n + \Delta t (u_t)_j^n + \frac{\Delta t^2}2 (u_{tt})_j^n + \mathcal{O}(\Delta t^3)
        - \alpha \left[
            u_j^n - \Delta x (u_x)_j^n + \frac{\Delta x^2}2 (u_{xx})_j^n + \mathcal{O}(\Delta x^3)
        \right]                                                                                                              \\
                              & - \beta u_j^n
        - \gamma \left[
            u_j^n + \Delta x (u_x)_j^n + \frac{\Delta x^2}2 (u_{xx})_j^n + \mathcal{O}(\Delta x^3)
        \right]                                                                                                              \\
        ={}                   & u_j^n - a \Delta t (u_x)_j^n + a^2 \frac{\Delta t^2}2 (u_{xx})_j^n + \mathcal{O}(\Delta t^3)
        - \alpha \left[
            u_j^n - \Delta x (u_x)_j^n + \frac{\Delta x^2}2 (u_{xx})_j^n + \mathcal{O}(\Delta x^3)
        \right]                                                                                                              \\
                              & - \beta u_j^n
        - \gamma \left[
            u_j^n + \Delta x (u_x)_j^n + \frac{\Delta x^2}2 (u_{xx})_j^n + \mathcal{O}(\Delta x^3)
        \right]                                                                                                              \\
        ={}                   & (1-\alpha - \beta - \gamma) u_j^n +
        (- a \Delta t + \alpha \Delta x - \gamma \Delta x) (u_x)_j^n
        + \left(a^2 \frac{\Delta t^2}2 - \alpha \frac{\Delta x^2}2 - \gamma \frac{\Delta x^2}2\right) (u_{xx})_j^n           \\
                              & + \mathcal{O}(\Delta t^3) + (|\alpha| + |\beta| + |\gamma|) \mathcal{O}(\Delta x^3)
    \end{align*}
    相容性要求参数满足如下方程组
    \[
        \left\{
        \begin{aligned}
            1-\alpha - \beta - \gamma ={}                                                      & 0, \\
            - a \Delta t + \alpha \Delta x - \gamma \Delta x ={}                               & 0, \\
            a^2 \frac{\Delta t^2}2 - \alpha \frac{\Delta x^2}2 - \gamma \frac{\Delta x^2}2 ={} & 0
        \end{aligned}
        \right.
        \quad \Rightarrow \quad
        \left\{
        \begin{aligned}
            \alpha ={} & \frac12 \left[\left(\frac{a \Delta t}{\Delta x}\right)^2 + \left(\frac{a \Delta t}{\Delta x}\right)\right], \\
            \beta ={}  & 1 - \left(\frac{a \Delta t}{\Delta x}\right)^2,                                                             \\
            \gamma ={} & \frac12 \left[\left(\frac{a \Delta t}{\Delta x}\right)^2 - \left(\frac{a \Delta t}{\Delta x}\right)\right]
        \end{aligned}
        \right.
    \]
    代入可知实际的格式为(记 $r = \frac{a \Delta t}{\Delta x}$)
    \begin{align*}
        v_j^{n+1} ={} & \frac12 (r^2 + r) v_{j-1}^n + (1-r^2) v_j^n + \frac12 (r^2 - r) v_{j+1}^n                   \\
        ={}           & v_j^n - \frac{r}2 (v_{j+1}^n - v_{j-1}^n) + \frac{r^2}2 (v_{j+1}^n - 2 v_j^n + v_{j-1}^{n})
    \end{align*}
    这就是 Lax-Wendroff 格式,对应的局部截断误差为 $\mathcal{O}(\Delta x^2 + \Delta t^2)$。
\end{solution*}


\section{变系数对流/扩散方程}

考虑变系数对流方程
\[
    u_t + a(x,t) u_x = 0
\]
特征线满足
\[
    \frac{d x(t)}{d t} = a(x(t),t)
\]
特征线为一族互不相交的曲线,解沿着特征线不变。

可以直接将常系数对流方程的格式直接推广到变系数对流方程,例如 Lax–Friedrichs 格式
\[
    \frac{v_j^{n+1} - \frac12(v_{j-1}^n + v_{j+1}^n)}{\Delta t} + a(x_j,t^n) \frac{v_{j+1}^n - v_{j-1}^n}{2\Delta x} = 0
\]
在分析稳定性时可以采用冻结系数法,将系数冻结为某个常数,视作常系数方程进行分析。


考虑变系数扩散方程
\[
    u_t = b(x,t) u_{xx},\quad b(x,t) \ge \varepsilon > 0.
\]
可以直接将常系数扩散方程的格式直接推广到变系数扩散方程。
对于守恒型扩散方程
\[
    u_t = (b(x,t) u_x)_x
\]
可以在格式构造时考虑方程的守恒性质,例如
\[
    \frac{v_{j}^{n+1} - v_j^n}{\Delta t} = \frac{1}{\Delta x}
    \left(
    b_{j+\frac12}^n \frac{v_{j+1}^n - v_j^n}{\Delta x} - b_{j-\frac12}^n \frac{v_{j}^n - v_{j-1}^n}{\Delta x}
    \right)
\]


\section{一阶线性偏微分方程组问题}

\subsection*{常系数一阶线性偏微分方程组}

考虑方程组
\begin{equation}
    U_t + A\, U_x = 0    \label{eq:1d-pde-system}
\end{equation}
其中 $A \in \mathbb{R}^{n \times n}$ 是常系数矩阵,未知函数 $U: \mathbb{R}\times \mathbb{R}^+ \to \mathbb{R}^n$。

可以直接将标量方程的格式推广到方程组,例如 Lax–Friedrichs 格式
\[
    V_j^{n+1} = \frac12(V_{j-1}^n + V_{j+1}^n) - \frac{\Delta t}{2\Delta x} A (V_{j+1}^n - V_{j-1}^n)
\]
Lax-Wendroff 格式
\[
    V_j^{n+1} = V_j^n - \frac{\Delta t}{2\Delta x} A (V_{j+1}^n - V_{j-1}^n)
    + \frac{\Delta t^2}{2\Delta x^2} A^2(V_{j+1}^n - 2 V_j^n + V_{j-1}^n)
\]

\begin{definition}
    称方程组~\eqref{eq:1d-pde-system} 为双曲型方程组,如果 $A$ 可以相似对角化,并且特征值均为实数。
    \[
        A = S \Lambda S^{-1},\quad
        \Lambda = \text{diag}(\lambda_1,\cdots,\lambda_n).\quad (\lambda_i \in \mathbb{R}).
    \]
    进一步,称方程组~\eqref{eq:1d-pde-system} 为严格双曲型方程组,如果 $A$ 是双曲型方程组,并且特征值互异。
\end{definition}

对于双曲型方程组,可以通过换元使得各个分量解耦:记 $W = S^{-1} U$,则 $W$ 满足
\[
    W_t + \Lambda W_x = 0
\]
记 $|\Lambda| = \text{diag}(|\lambda_1|,\cdots,|\lambda_n|)$,
可以得到关于 $W$ 的迎风格式
\[
    W_j^{n+1} = W_j^n
    - \frac{\Delta t}{2\Delta x} \Lambda (W_{j+1}^n - W_{j-1}^n)
    + \frac{\Delta t}{2\Delta x} |\Lambda|(W_{j+1}^n - 2 W_j^n + W_{j-1}^n)
\]
如果定义 $|A| = S |\Lambda| S^{-1}$,可以等价地改写为
\[
    V_j^{n+1} = V_j^n
    - \frac{\Delta t}{2\Delta x} A (V_{j+1}^n - V_{j-1}^n)
    + \frac{\Delta t}{2\Delta x} |A| (V_{j+1}^n - 2 V_j^n + V_{j-1}^n)
\]

\subsection*{变系数一阶线性偏微分方程组}

考虑方程组
\begin{equation}
    U_t + A(x,t) U_x = 0    \label{eq:1d-pde-system-var}
\end{equation}
其中 $A(x,t) \in \mathbb{R}^{n \times n}$,未知函数 $U: \mathbb{R}\times \mathbb{R}^+ \to \mathbb{R}^n$。


\begin{definition}
    称方程组~\eqref{eq:1d-pde-system-var} 为双曲型方程组,如果 $A(x,t)$ 可以相似对角化,并且特征值均为实数。
    \[
        A(x,t) = S(x,t) \Lambda(x,t) S(x,t)^{-1},\quad
        \Lambda(x,t) = \text{diag}(\lambda_1(x,t),\cdots,\lambda_n(x,t)).\quad (\lambda_i(x,t) \in \mathbb{R})
    \]
    进一步,称方程组~\eqref{eq:1d-pde-system-var} 为严格双曲型方程组,如果 $A(x,t)$ 是双曲型方程组,并且特征值始终互异。
\end{definition}

可以将常系数方程组的格式推广到变系数问题中。
